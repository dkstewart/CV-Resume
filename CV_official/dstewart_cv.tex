%% start of file `template.tex'.
%% Copyright 2006-2015 Xavier Danaux (xdanaux@gmail.com).
%
% This work may be distributed and/or modified under the
% conditions of the LaTeX Project Public License version 1.3c,
% available at http://www.latex-project.org/lppl/.


\documentclass[11pt,a4paper,sans]{moderncv}        % possible options include font size ('10pt', '11pt' and '12pt'), paper size ('a4paper', 'letterpaper', 'a5paper', 'legalpaper', 'executivepaper' and 'landscape') and font family ('sans' and 'roman')

% moderncv themes
\moderncvstyle{classic}

                           % style options are 'casual' (default), 'classic', 'banking', 'oldstyle' and 'fancy'
\moderncvcolor{blue}                               % color options 'black', 'blue' (default), 'burgundy', 'green', 'grey', 'orange', 'purple' and 'red'
%\renewcommand{\familydefault}{\sfdefault}         % to set the default font; use '\sfdefault' for the default sans serif font, '\rmdefault' for the default roman one, or any tex font name
%\nopagenumbers{}                                  % uncomment to suppress automatic page numbering for CVs longer than one page

% character encoding
%\usepackage[utf8]{inputenc}                       % if you are not using xelatex ou lualatex, replace by the encoding you are using
%\usepackage{CJKutf8}                              % if you need to use CJK to typeset your resume in Chinese, Japanese or Korean

% adjust the page margins
\usepackage[scale=0.75]{geometry}
\usepackage{bibentry}
%\setlength{\hintscolumnwidth}{3cm}                % if you want to change the width of the column with the dates
%\setlength{\makecvtitlenamewidth}{10cm}           % for the 'classic' style, if you want to force the width allocated to your name and avoid line breaks. be careful though, the length is normally calculated to avoid any overlap with your personal info; use this at your own typographical risks...


% personal data
\name{Danielle}{Stewart}
\title{CV}                               % optional, remove / comment the line if not wanted
%\address{6-248 Keller Hall. 200 Union St. S.E.}{Minneapolis, MN 55455}{USA}% optional, remove / comment the line if not wanted; the "postcode city" and "country" arguments can be omitted or provided empty
%\phone[mobile]{+1~(218)~349~5155}                   % optional, remove / comment the line if not wanted; the optional "type" of the phone can be "mobile" (default), "fixed" or "fax"
%\phone[fixed]{+2~(345)~678~901}
%\phone[fax]{+3~(456)~789~012}
\email{dkstewar@umn.edu}                               % optional, remove / comment the line if not wanted
\homepage{www-users.cs.umn.edu/\textasciitilde dkstewar/}                      % optional, remove / comment the line if not wanted
\social[linkedin]{daniellestewart}                        % optional, remove / comment the line if not wanted
%\social[twitter]{jdoe}                             % optional, remove / comment the line if not wanted
\social[github]{dkstewart}                              % optional, remove / comment the line if not wanted
%\extrainfo{additional information}                 % optional, remove / comment the line if not wanted
%\photo[64pt][0.4pt]{picture}                       % optional, remove / comment the line if not wanted; '64pt' is the height the picture must be resized to, 0.4pt is the thickness of the frame around it (put it to 0pt for no frame) and 'picture' is the name of the picture file
%\quote{Some quote}                                 % optional, remove / comment the line if not wanted

% bibliography adjustements (only useful if you make citations in your resume, or print a list of publications using BibTeX)
%   to show numerical labels in the bibliography (default is to show no labels)
\makeatletter\renewcommand*{\bibliographyitemlabel}{\@biblabel{\arabic{enumiv}}}\makeatother
%   to redefine the bibliography heading string ("Publications")
%\renewcommand{\refname}{Articles}

% bibliography with mutiple entries
%\usepackage{multibib}
%\newcites{book,misc}{{Books},{Others}}
%----------------------------------------------------------------------------------
%            content
%----------------------------------------------------------------------------------
\begin{document}
%\begin{CJK*}{UTF8}{gbsn}                          % to typeset your resume in Chinese using CJK
%-----       resume       ---------------------------------------------------------
\makecvtitle

\section{Education}
\cventry{2016--Present}{Ph.D. Candidate in Computer Science}{University of Minnesota - Twin Cities}{Minneapolis, MN, USA}{Critical Systems Research Group}{Advisors: Dr. M. Heimdahl and Dr. M. W. Whalen}


\cventry{2013--2015}{Master in Mathematics}{University of Minnesota, Duluth}{Duluth, MN, USA}
{Thesis: Even Harmonious Labelings of Disconnected Graphs}{Advisor: Dr. J. Gallian}
%\cvitem{description}{Short thesis abstract}


\cventry{2011--2013}{Bachelor in Mathematics}{University of Minnesota, Duluth}{Duluth, MN, USA}
{Thesis: Generation of Pseudoprimes}{Advisor: Dr. J. Greene}


%\cvitem{description}{Short thesis abstract}

\cventry{2008--2011}{Associate of Arts}{Lake Superior College}{Duluth, MN, USA}{}{}


\cventry{1998--2002}{High School Diploma}{Bemidji High School (Homeschooled)}{Bemidji, MN, USA}{}{}


\section{Research Interests}
Safety analysis of systems, model based safety analysis, software verification, formal methods, model checking, dependable and secure software development, software testing.

\section{Work Experience}
\cventry{Jul. 2018-- Present}{Visiting Researcher}{Deutsches Zentrum für Luft und Raumfahrt (DLR)}{}{}{Model-based safety analysis applied to an unmanned aerospace vehicle project.}

\cventry{Dec. 2017-- Present}{Formal Methods Consulting}{Stottler Henke Associates, Inc}{}{}{Compositional analysis verification in critical systems aviation projects.}

\cventry{Sept. 2017-- Sept. 2018}{Course Development}{Coursera: Software Engineering}{}{}{Assisted in course organization, exams, quizzes, and other course development activities.}

\cventry{Dec. 2016-- Present}{Research Assistant}{Critical Systems Group, University of Minnesota}{}{}{Research in safety analysis for the NASA AMASE (Architectural Modeling and Analysis for Safety Engineering) project.}

\cventry{Aug. 2015-- May. 2016}{Instructor}{University of Minnesota, Duluth: Dept. of Maths}{}{}{Instructor for Differential Equations, College Algebra, and Algebra I-II.}

\cventry{Aug. 2013-- May. 2015}{Teaching Assistant}{University of Minnesota, Duluth: Dept. of Maths}{}{}{Teaching assistant for Elementary Real Analysis, Calculus II, and Approximation \& Quadrature.}


 \newpage
\nocite{*}
\bibliographystyle{unsrt}
\bibliography{publications}
  %\nobibliography{publications}
 % \bibliographystyle{unsrt}


\vspace{0.1in}

\section{\textbf{Presentations}}
\begin{itemize}
\item AMASE Second Year End Review, NASA, Oct. 2018, Langley, VA, USA
\item Model Based Safety Analysis, DLR, Jul. 2018, Braunschweig, Germany
\item Critical Systems Research, Code Freeze, Jan. 2018, Minneapolis, MN, USA
\item AMASE First Year End Review, NASA, Sept. 2017, Langley, VA, USA
\item Architectural Modeling and Analysis for Safety Engineering, IMBSA, Sept. 2017, Trento, Italy
\item  Properly Even Harmonious Graphs, IWOCA, Oct. 2014, Duluth, MN, USA
\end{itemize}



\section{Honors and Awards}
\cvitem {2016} {Awarded College of Science and Engineering Graduate Fellowship, University of Minnesota}
\cvitem {2015} {SCSE Outstanding Teaching Assistant Award, University of Minnesota, Duluth}
\cvitem {2016} {UMD Mathematics Departmental Teaching Assistant Award, University of Minnesota, Duluth}
\cvitem {2014} {Summer Research Fellowship, Dept. of Mathematics, University of Minnesota, Duluth}
\cvitem{2013}{Undergraduate Research Opportunities Grant, University of Minnesota, Duluth}
\cvitem{2013}{Duane E. Anderson Memorial Fellowship, University of Minnesota, Duluth}
\cvitem {2012--2014}{Pi Mu Epsilon Honor Society, University of Minnesota, Duluth: Dept. of Mathematics}
\cvitem {2011--2012}{Martha Lahti Scholarship, University of Minnesota, Duluth}
%\cvitem {2010}{Student of the Year Award, Lake Superior College, Duluth, MN}
\cvitem {2009}{Student of the Year, Biology Dept. Award, Lake Superior College, Duluth, MN}

\newpage
\section{Professional Activities}
\subsection{\textbf{Peer Reviewer}}
\begin{itemize}
\item MEMOCODE 2018: 16th International Conference on Formal Methods and Models for System Design
\item FM 2018: International Symposium on Formal Methods
\item NFM 2018: 10th NASA Formal Methods Symposium
\item ASE 2017: 32nd IEEE/ACM International Conference on Automated Software Engineering
\item SETTA 2017: 3rd Symposium on Dependable Software Engineering
\item MEMOCODE 2017: 15th International Conference on Formal Methods and Models for System Design
\end{itemize}
\vspace{0.1in}

\subsection{\textbf{Service}}
\begin{itemize}
\item  Graduate Council Student Representative, University of Minnesota, Duluth: 2014-2015
\end{itemize}



%\section{Languages}
%\cvitemwithcomment{Language 1}{Skill level}{Comment}
%\cvitemwithcomment{Language 2}{Skill level}{Comment}
%\cvitemwithcomment{Language 3}{Skill level}{Comment}

\section{Selected Course Projects}
\begin{itemize}
\item Sequent Calculus Proof Checker (OCaml)
    \begin{itemize}
    \item Topics in Computation and Deduction, 2016
    \end{itemize}

%\item Device Driver for Linux OS
%    \begin{itemize}
%    \item Operating Systems Course, 2016
%    \end{itemize}

\item Phishing Detection Using Natural Language Processing Techniques
    \begin{itemize}
    \item Computer Security Course, 2016
    \end{itemize}
    
\item Lexer, Parser, Evaluator, and Type-Checker for Imperative Language in OCaml
    \begin{itemize}
    \item Programming Languages Course, 2014
    \end{itemize}



\end{itemize}

\section{Computer skills}
\cvitem{Programming}{Java, Perl, OCaml, Python, C++, LaTex, Prolog, MIPS Assembly}
\cvitem{Modeling}{AADL, Lustre}
\cvitem{Tools}{AGREE, Safety Annex for AADL, SOTERIA}

\section{Languages}
\cvitem{English}{Native Speaker}
\cvitem{German}{Working Knowledge}
%\cvitem{hobby 3}{Description}
%
%\section{Extra 2}
%\cvlistdoubleitem{Item 1}{Item 4}
%\cvlistdoubleitem{Item 2}{Item 5\cite{book1}}
%\cvlistdoubleitem{Item 3}{Item 6. Like item 3 in the single column list before, this item is particularly long to wrap over several lines.}

\section{References}
\begin{itemize}


\item Mats Heimdahl
    \begin{itemize}
    \item Department of Computer Science \& Engineering, University of Minnesota, MN, USA.

    \cvlistdoubleitem{Email: heimdahl@cs.umn.edu}{Tel: +1-612-625-2068}
    \end{itemize}
    
    \item Michael W. Whalen
    \begin{itemize}
    \item Department of Computer Science \& Engineering, University of Minnesota, MN, USA.

    \cvlistdoubleitem{Email: whalen@cs.umn.edu}{Tel: +1-612-624-5130}
    \end{itemize}

\item Joseph Gallian
    \begin{itemize}
    \item Deptartment of Mathematics, University of Minnesota Duluth, MN, USA.

    \cvlistdoubleitem{Email: jgallian@d.umn.edu}{Tel: +1~(218)~726~7576}
    \end{itemize}
\end{itemize}

% Publications from a BibTeX file without multibib
%  for numerical labels: \renewcommand{\bibliographyitemlabel}{\@biblabel{\arabic{enumiv}}}% CONSIDER MERGING WITH PREAMBLE PART
%  to redefine the heading string ("Publications"): \renewcommand{\refname}{Articles}
%\nocite{*}
%\bibliographystyle{plain}
%\bibliography{publications}                        % 'publications' is the name of a BibTeX file

% Publications from a BibTeX file using the multibib package
%\section{Publications}
%\nocitebook{book1,book2}
%\bibliographystylebook{plain}
%\bibliographybook{publications}                   % 'publications' is the name of a BibTeX file
%\nocitemisc{misc1,misc2,misc3}
%\bibliographystylemisc{plain}
%\bibliographymisc{publications}                   % 'publications' is the name of a BibTeX file

%\clearpage
%%-----       letter       ---------------------------------------------------------
%% recipient data
%\recipient{Company Recruitment team}{Company, Inc.\\123 somestreet\\some city}
%\date{January 01, 1984}
%\opening{Dear Sir or Madam,}
%\closing{Yours faithfully,}
%\enclosure[Attached]{curriculum vit\ae{}}          % use an optional argument to use a string other than "Enclosure", or redefine \enclname
%\makelettertitle
%
%Lorem ipsum dolor sit amet, consectetur adipiscing elit. Duis ullamcorper neque sit amet lectus facilisis sed luctus nisl iaculis. Vivamus at neque arcu, sed tempor quam. Curabitur pharetra tincidunt tincidunt. Morbi volutpat feugiat mauris, quis tempor neque vehicula volutpat. Duis tristique justo vel massa fermentum accumsan. Mauris ante elit, feugiat vestibulum tempor eget, eleifend ac ipsum. Donec scelerisque lobortis ipsum eu vestibulum. Pellentesque vel massa at felis accumsan rhoncus.
%
%Suspendisse commodo, massa eu congue tincidunt, elit mauris pellentesque orci, cursus tempor odio nisl euismod augue. Aliquam adipiscing nibh ut odio sodales et pulvinar tortor laoreet. Mauris a accumsan ligula. Class aptent taciti sociosqu ad litora torquent per conubia nostra, per inceptos himenaeos. Suspendisse vulputate sem vehicula ipsum varius nec tempus dui dapibus. Phasellus et est urna, ut auctor erat. Sed tincidunt odio id odio aliquam mattis. Donec sapien nulla, feugiat eget adipiscing sit amet, lacinia ut dolor. Phasellus tincidunt, leo a fringilla consectetur, felis diam aliquam urna, vitae aliquet lectus orci nec velit. Vivamus dapibus varius blandit.
%
%Duis sit amet magna ante, at sodales diam. Aenean consectetur porta risus et sagittis. Ut interdum, enim varius pellentesque tincidunt, magna libero sodales tortor, ut fermentum nunc metus a ante. Vivamus odio leo, tincidunt eu luctus ut, sollicitudin sit amet metus. Nunc sed orci lectus. Ut sodales magna sed velit volutpat sit amet pulvinar diam venenatis.
%
%Albert Einstein discovered that $e=mc^2$ in 1905.
%
%\[ e=\lim_{n \to \infty} \left(1+\frac{1}{n}\right)^n \]
%
%\makeletterclosing

%\clearpage\end{CJK*}                              % if you are typesetting your resume in Chinese using CJK; the \clearpage is required for fancyhdr to work correctly with CJK, though it kills the page numbering by making \lastpage undefined
\end{document}


%% end of file `template.tex'.
